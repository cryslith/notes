\documentclass{article}
\usepackage[utf8]{inputenc}
\usepackage[margin=1in]{geometry}
\input{liatex/preamble}

\usepackage{tikz}
\usepackage{tcolorbox}
\usepackage{lipsum}

\newcommand{\example}[2]{
  \begin{minipage}[t]{\textwidth/2-1em}   
    \textbf{Example problem:}\\
    \itshape
    {#1}
  \end{minipage}\hfill
  \begin{minipage}[t]{\textwidth/2-1em}
    \textbf{Example proof:}\\
    {#2}
  \end{minipage}
}

\newtcolorbox{logicbox}[1]{colbacktitle=red!40!white, coltitle=black, title={How to prove {#1}}}

\title{How to prove it}
\author{Lily Chung}
\date{}

\begin{document}
\maketitle

\begin{tcolorbox}[title=Start]
  Write down the problem statement, including \textbf{all} assumptions.
  \\[1em]
  \example{
    Suppose $p$ is a prime and $a$ is not a
    multiple of $p$. Prove that $a^{p-1} \equiv 1 \pmod p$.
  }{
    Assume $p$ is a prime.
    Assume $a$ is not a multiple of $p$.
    We want to show $a^{p-1} \equiv 1 \pmod p$. [Proof.]
  }
\end{tcolorbox}

\begin{logicbox}{$P \to Q$}
  Assume $P$ and prove $Q$.
  \\[1em]
  \example{
    Prove $(3a = 3b) \to (a = b)$ for real numbers $a, b$.
  }{
    Assume $3a = 3b$.
    We want to show $a = b$. [Proof.]
  }
\end{logicbox}

\begin{logicbox}{$P \land Q$}
  First prove $P$, then prove $Q$.
\end{logicbox}

\begin{logicbox}{$P \lor Q$}
  Choose one of $P$ or $Q$ to prove, then prove it.
\end{logicbox}

\begin{logicbox}{$\neg P$}
  Assume $P$ and prove a contradiction.
  \\[1em]
  \example{
    Prove that there is no integer $n$ such that $n^2 = -1$.
  }{
    Assume for contradiction that there does exist $n$ such that $n^2 = -1$.
    [Proof of contradiction.]
    We have arrived at a contradiction, so no such integer $n$ exists.
  }
\end{logicbox}

\begin{logicbox}{$P \iff Q$}
  This is the same as $(P \to Q) \land (Q \to P)$.
  First prove $P \to Q$, then prove $Q \to P$.
  \\[1em]
  \example{
    Prove that $x$ is divisible by 9 if and
only if the sum of the digits of $x$ is a multiple of 9.
  }{
    First, we assume $x$ is divisible by 9, and prove that
    the sum of the digits of $x$ is a multiple of 9.
    [Proof.]
    \\[0.5em]
    Now, we assume the sum of the digits of $x$ is a multiple of 9,
    and prove that $x$ is divisible by 9. [Proof.]
  }
\end{logicbox}

\end{document}
