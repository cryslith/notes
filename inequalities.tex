\documentclass{article}
\usepackage[utf8]{inputenc}
\usepackage[margin=1in]{geometry}
\input{liatex/preamble}

\title{Common Inequalities}
\author{Lily Chung}
\date{}

\begin{document}
\maketitle

Some nice derivations of common inequalities.

\section*{Power Means}

Let $X$ be a nonnegative random variable, and define:

\[M_p(X) =
\begin{cases}
  \E[X^p]^{1/p} & p \in \R - \{0\} \\
  \exp \E[\log X] & p = 0 \\
  \essinf(X) & p = -\infty \\
  \esssup(X) & p = \infty \\
\end{cases}\]

\begin{theorem}[Power Mean Inequality]
  If $p \le q$ then $M_p(X) \le M_q(X)$ whenever both exist.
\end{theorem}
\begin{proof}
  If either of $p$ or $q$ is infinite then the result is trivial.

  Suppose $0 = p < q$.  Since $\log$ is concave we have by Jensen's inequality
  \[
  \log \E[X^p] &\ge \E[\log X^p] \\
  \frac1p \log \E[X^p] &\ge \E[\log X] \\
  \E[X^p]^{1/p} &\ge \exp \E[\log X] \\
  \]
  The case where $p < q = 0$ is analogous.

  Now suppose $0 < p < q$.  Since $u \mapsto u^{q/p}$ is convex we have
  \[
  \E[X^p]^{q/p} &\le \E[X^q] \\
  \E[X^p]^{1/p} &\le \E[X^q]^{1/q} \\
  \]
  The case where $p < q < 0$ is analogous.
\end{proof}

% xxx need to prove existence of M_0 so there's no issue with transitivity
% also maybe clarify 1/0 = +inf in this context

For instance this generalizes the AM-GM inequality since $M_0$ is the geometric mean and $M_1$ is the arithmetic mean.

\section*{H\"older's Inequality}

Let $f : S \to \R$ be a nonnegative measurable function.
Define \[\norm{f}_p = \begin{cases}
  \left(\int f\,d\mu\right)^{1/p} & 1 \le p < \infty \\
  \esssup f & p = \infty \\
  \end{cases}
\]
Note that the results from the previous section do not directly apply here since $S$ is not necessarily a probability space.  Nonetheless we can make use of them by defining an appropriate probability measure.

\begin{theorem}[H\"older's Inequality]
  Let $f, g : S \to \R$ be nonnegative measurable functions,
  and let $p, q \in [1, \infty]$ such that $\frac1p + \frac1q = 1$.
  Then $\norm{fg}_1 \le \norm{f}_p \norm{g}_q$.
\end{theorem}
\begin{proof}
  The case where $p$ or $q$ is $\infty$ is trivial, so we skip it.
  Define $\nu$ to be a probability measure on $S$ via $\nu(A) = \frac{\int_A g^q\,d\mu}{\int g^q\,d\mu}$,
  and let $X = fg^{1-q}$.  (We can assume $\int g^q\,d\mu \in (0, \infty)$ since otherwise the result is trivial.)
  Then we have
  \[\norm{fg}_1 &= \left(\int g^q\,d\mu\right)\left(\int f g^{1-q} \frac{g^q}{\int g^q\,d\mu}\,d\mu\right) \\
  &= \left(\int g^q\,d\mu\right)\E_\nu[X] \\
  &\le \left(\int g^q\,d\mu\right)\E_\nu[X^p]^{1/p} \\
  &= \left(\int g^q\,d\mu\right)\left(\int f^p g^{p(1-q)}\frac{g^q}{\int g^q\,d\mu}\,d\mu \right)^{1/p} \\
  &= \left(\int g^q\,d\mu\right)^{1 - 1/p}\left(\int f^p g^{p + q - pq}\,d\mu \right)^{1/p} \\
  &= \norm{f}_p \norm{g}_q \\
\]
\end{proof}

\end{document}
