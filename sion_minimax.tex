\documentclass{article}
\usepackage[utf8]{inputenc}
\usepackage[margin=1in]{geometry}
\input{liatex/preamble}

\usepackage[style=alphabetic]{biblatex}
\addbibresource{notes.bib}

\newcommand*{\Prob}[1]{\mathcal{P}({#1})}
\newcommand*{\Probstar}[1]{\mathcal{P}^*({#1})}
\newcommand*{\Probfin}[1]{\Delta_{#1}}
\DeclareMathOperator{\cvx}{cvx}
\DeclareMathOperator{\cl}{cl}
\DeclareMathOperator{\im}{im}

\title{Sion minimax notes}
\author{Lily Chung}
\date{}

\begin{document}
\maketitle

Notes accompanying Maurice Sion, ``On General Minimax Theorems'' \cite{Sion1958}.

\section*{Semi-finite Von Neumann minimax}

This material is based on \cite[Sections 2.3--2.4]{Blackwell1954} and \cite[Section 13.1]{Ferguson2020}

In this section, we will be interested in mixed strategies for zero-sum games where one player has only finitely many options.\footnote{These are called ``S-games'' in \cite{Blackwell1954}.}
Since the other player has infinitely many options,
we will assume these live in a measure space so that we can
represent their mixed strategies as probability distributions.

Let $M = [m]$ be a finite set, $N$ be an arbitrary measurable space,
and $f : M \times N \to \R$ be measurable.
Together these define the game $(M, N, f)$.
The maximizer will choose a mixed strategy from $\Probfin{M}$, the set of probability distributions on the finite set $M$.
However, we will require the minimizer to choose their mixed strategy from $\Probstar{N}$, which we define as the set of probability distributions $P_Y$ on $N$ with the property that $\E[f(X, Y)]$ exists for all pairs of independent random variables $X, Y$ where $Y \sim P_Y$.\footnote{Equivalently, $\E[f(x, Y)]$ exists for all $x \in M$, and we do not have $\E[f(x_1, Y)] = \infty$ and $\E[f(x_2, Y)] = -\infty$ simultaneously for $x_1, x_2 \in M$.}
Of course, this condition is satisfied if $f$ is bounded below or bounded above.

We write the minimax inequality for mixed strategies on $(M, N, f)$ as
\[\sup_{P_X \in \Probfin{M}} \inf_{P_Y \in \Probstar{N}} \E[f(X, Y)] \le \inf_{P_Y \in \Probstar{N}} \sup_{P_X \in \Probfin{M}} \E[f(X, Y)]\]
with the understanding that $X$ and $Y$ are sampled independently from $P_X$ and $P_Y$.

We can view $f$ in the equivalent form $\tilde f : N \to \R^m$.
Then if $X$ and $Y$ are independent random variables on $M$ and $N$ such that $\E[f(x, Y)]$ is finite for every $x$, we have
\[
\E[f(X, Y)] &= \E[(x \mapsto \E[f(x, Y)])(X)] &\text{by independence} \\
&= \E[(x \mapsto \E[\tilde f(Y)]_x)(X)] \\
&= \E[\E[\tilde f(Y)]_X] \\
\]

So the outcome of the game depends only on $\E[\tilde f(Y)]$.\footnote{%
Incidentally, it can be shown that if $Z$ is a random variable supported on $A \subset \R^n$ with finite expectation, then $\E Z \in \cvx(A)$.
See \url{https://math.stackexchange.com/questions/284153/mathematical-expectation-is-inside-convex-hull-of-support}.}

Now we show that the minimax equality holds.

\begin{theorem}[Semi-finite Von Neumann minimax, arbitrary distributions]\label{thm:semi-finite}
  Let $M = [m]$ be a finite set, $N$ be a measurable space,
  and $f : M \times N \to \R$ be measurable.
  Define
  \[
  c^* \eqdef \inf\{c \in \R : (-\infty, c]^m \cap \cvx(S) \ne \emptyset\},\]
  where $S = \im \tilde f \subset \R^m$.
  Then
  \[c^* = \sup_{P_X \in \Probfin{M}} \inf_{P_Y \in \Probstar{N}} \E[f(X, Y)] = \inf_{P_Y \in \Probstar{N}} \sup_{P_X \in \Probfin{M}} \E[f(X, Y)].\]
\end{theorem}
\begin{proof}
  It suffices to show the inequalities
  \begin{align}
    c^* &\le \sup_{P_X \in \Probfin{M}} \inf_{P_Y \in \Probstar{N}} \E[f(X, Y)] \label{eqn:semi-finite:lb} \\
    c^* &\ge \inf_{P_Y \in \Probstar{N}} \sup_{P_X \in \Probfin{M}} \E[f(X, Y)] \label{eqn:semi-finite:ub}
  \end{align}

  First we show (\ref{eqn:semi-finite:lb}).
  Fix an arbitrary $c < c^*$, so that $(-\infty, c]^m \cap \cvx(S) = \emptyset$.
  By the hyperplane separation theorem, there exist $\mu \in \R^m - \{0\}$ and $\beta \in \R$ such that $\mu^T w \le \beta \le \mu^T v$ for all $w \in (-\infty, c]^m$ and $v \in \cvx(S)$.
  Each $\mu_i$ must be nonnegative, and by rescaling we can assume $\mu \in \Probfin{M}$.
  Then we can compute:
  \[
  \sup_{P_X \in \Probfin{M}} \inf_{P_Y \in \Probstar{N}} \E[f(X, Y)]
  &\ge \inf_{P_Y \in \Probstar{N}} \E_{X \sim \mu, Y \sim P_Y}[f(X, Y)] \\
  &= \inf_{P_Y \in \Probstar{N}} \E[\mu^\tp \tilde f(Y)] \\
%  &\ge \beta &\text{since $\tilde f(Y) \in \cvx(S)$.} \\
  &\ge \mu^\tp (c, c, \dots, c) &\text{since $\tilde f(Y) \in \cvx(S)$.} \\
  &= c
  \]
  which entails (\ref{eqn:semi-finite:lb}).

  Now we show (\ref{eqn:semi-finite:ub}).
  Fix an arbitrary $c > c^*$ so that $(-\infty, c]^m \cap \cvx(S) \ne \emptyset$, which is to say there exists $z \in \cvx(S)$ such that each $z_i \le c$.
  Since $z \in \cvx(S)$ there exists a probability measure $\nu$ on $N$
  such that $\E_{Y \sim \nu}[\tilde f(Y)] = z$.
  To see this, write $z$ as a convex combination $\sum_i \lambda_i \tilde f(y_i)$,
  where each $y_i$ is in $N$ and the $\tilde f(y_i)$ are distinct.
  Then define the probability distribution $\nu$ on $N$ by $\nu[A] \eqdef \sum_{i : y_i \in A} \lambda_i$.

  Now we compute:
  \[
  \inf_{P_Y \in \Probstar{N}} \sup_{P_X \in \Probfin{M}} \E[f(X, Y)]
  &\le \sup_{P_X \in \Probfin{M}} \E_{X \sim P_X, Y \sim \nu}[f(X, Y)] \\
  &= \sup_{P_X \in \Probfin{M}} \E_{X \sim P_X}[\E_{Y \sim \nu}[\tilde f(Y)]_X] \\
  &= \sup_{P_X \in \Probfin{M}} \E_{X \sim P_X}[z_X] \\
  &\le c
  \]
  and so (\ref{eqn:semi-finite:ub}) holds, which concludes the proof.
\end{proof}

The proof of \cref{thm:semi-finite} shows that allowing arbitrary random variables $Y$ was unnecessary.
The minimax value is unchanged if we restrict the minimizer's mixed strategy to simply be a discrete probability distribution taking values in a finite subset of $N$.
This version is conceptually simpler since we don't need $N$ to be a measurable space.

\begin{theorem}[Semi-finite Von Neumann minimax, finitely supported distributions]\label{thm:semi-finite delta}
  Let $M$ be a finite set, $N$ be an arbitrary set, and $f : M \times N \to \R$.
  Define $S$ and $c^*$ as in \cref{thm:semi-finite}.
  Then
  \[c^* = \sup_{P_X \in \Probfin{M}} \inf_{P_Y \in \Probfin{N}} \E[f(X, Y)] = \inf_{P_Y \in \Probfin{N}} \sup_{P_X \in \Probfin{M}} \E[f(X, Y)],\]
  where $\Probfin{N}$ is the set of finitely supported probability distributions on $N$.
\end{theorem}

In fact, it suffices to consider probability distributions supported on at most $m$ elements of $N$.
This is because by Caratheodory's theorem, any element $z \in \cvx(S)$ can be written as a convex combination of $m+1$ elements $s_1, \dots, s_{m+1} \in S$.
Since the function $g(v) = \max_{i \in [m]} v_i$ is continuous, convex, and unbounded,
it follows that $g$ attains a minimum at some point on the boundary of $\cvx(\{s_1, \dots, s_{m+1}\})$.
Such a boundary point can be written as a convex combination of at most $m$ of the elements $s_1, \dots, s_{m+1}$, and it can be used in place of $z$ in the proof of (\ref{eqn:semi-finite:ub}).

\section*{Concavelike-convexlike minimax}

\begin{definition}[Concavelike and convexlike, Definitions 2.1--2.3 in \cite{Sion1958}]
  Let $f : M \times N \to \R$.
  We say $f$ is \defn{concavelike} in $M$ if
  for all $\mu_0, \mu_1 \in M$ and $\lambda \in [0, 1]$,
  there exists $\tilde\mu \in M$ such that
  \[\overline{\lambda} f(\mu_0, y) + \lambda f(\mu_1, y) &\le f(\tilde\mu, y) &\forall y \in N.\]
  Similarly, $f$ is \defn{convexlike} in $N$ if
  for all $\nu_0, \nu_1 \in N$ and $\lambda \in [0, 1]$,
  there exists $\tilde\nu \in N$ such that
  \[\overline{\lambda} f(x, \nu_0) + \lambda f(x, \nu_1) &\ge f(x, \tilde\nu) &\forall x \in M.\]
  Finally, $f$ is \defn{concave-convexlike} if it is concavelike in $M$ and convexlike in $N$.
\end{definition}

An equivalent definition of concavelike is that for any distribution $\mu \in \Probfin{M}$, there exists $\tilde\mu \in M$ such that $\E_{X \sim \mu}[f(X, y)] \le f(\tilde\mu, y)$.

We now prove the following theorem, whose proof Sion omits.
\begin{theorem}[Theorem 4.1 in \cite{Sion1958}]
  Let $f : M \times N \to \R$ be concave-convexlike.
  Suppose that for any $c < \inf_N \sup_M f$ there exists a finite subset $T \subset M$
  such that $\inf_N \sup_T f > c$.
  Then $\inf_N \sup_M f = \sup_M \inf_N f$.
\end{theorem}
\begin{proof}
  Let $c < \inf_N \sup_M f$, so that $T$ exists as in the assumption.
  Then
  \[
  c &< \inf_N \sup_T f \\
  &= \inf_{P_Y \in \Probfin{N}} \sup_{x \in T} \E[f(x, Y)] &\text{since $f$ is convexlike} \\
  &\le \inf_{P_Y \in \Probfin{N}} \sup_{P_X \in \Probfin{T}} \E[f(X, Y)] \\
  &= \sup_{P_X \in \Probfin{T}} \inf_{P_Y \in \Probfin{N}} \E[f(X, Y)] &\text{\cref{thm:semi-finite delta}} \\  
  &\le \sup_{P_X \in \Probfin{M}} \inf_{P_Y \in \Probfin{N}} \E[f(X, Y)] \\
  &= \sup_M \inf_N f &\text{since $f$ is concave-convexlike} \\
  \]
\end{proof}

\printbibliography

\end{document}
