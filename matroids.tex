\documentclass{article}
\usepackage[utf8]{inputenc}
\usepackage[margin=1in]{geometry}
\input{liatex/preamble}

\newcommand*{\I}[0]{\mathcal{I}}
\newcommand*{\Cc}[0]{\mathcal{C}}

\title{Notes on Matroid Algorithms}
\author{Lily Chung}
\date{}

\begin{document}
\maketitle

\section*{Matroids}

todo define matroids and show some lemmas

%% In the following $M$ is a matroid, $E$ is its ground set, $\I$ are the independent sets, $\Cc$ are the circuits.
I sometimes write set union as $A + B$ when $A, B$ are disjoint, and I write set difference as $C - D$ when $D \subset C$.
I also sometimes write $A + a$ instead of $A + \{a\}$ and $C - d$ instead of $C - \{d\}$.

\section*{Matroid Partitioning}

\href{https://page.math.tu-berlin.de/\~felsner/Lehre/SemMatS/Literatur/ScheinermanUllman:FractionalArboricity.pdf}{based on book chapter}.

Let $\I_1, \dots, \I_n$ be matroid independence systems over the same ground set $E$.

A \defn{proper coloring} $S$ of $E$ with respect to $\I_1, \dots, \I_n$ is a partition of $E$ into sets $S_1, \dots, S_n$ such that each $S_i \in \I_i$.
The goal of matroid partitioning is to either find a proper coloring, or a certificate proving that no proper coloring exists.

The algorithm works by maintaining a \defn{partial proper coloring},
a collection of disjoint sets $S_1, \dots, S_n$ such that $S_i \in \I_i$ which don't necessarily cover all of $E$.
Initially the elements of $E$ are uncolored, so the sets $S_i$ are empty.
The algorithm repeatedly finds an \emph{augmenting path}
which it uses to color an element of $E$,
maintaining the invariant that the partial coloring is proper.
This continues until no augmenting path exists.

\begin{definition}
  Let $S$ be a partial proper coloring.
  We write $x \gets \langle i \rangle$ if $x$ doesn't have color $i$, and $S_i + x \in \I_i$.
  We also write $x \gets y$ if $y$ has color $i$, $x$ doesn't have color $i$, and $S_i - y + x \in \I_i$.
  An \defn{augmenting path} is a directed path of the form
  \[x_0 \gets x_1 \gets x_2 \gets \cdots \gets x_k \gets \langle a \rangle\]
  such that $x_0$ is uncolored, $x_i \not\gets x_j$ for $j > i + 1$, and $x_i \not\gets \langle a \rangle$ for $i < k$.
\end{definition}

When the algorithm finds an augmenting path, it shifts all the colors one step along the path to the left.
The following lemma proves that this preserves the invariant:

\begin{lemma}\label{lem:augmenting path}
  Let $S$ be a partial proper coloring,
  and let
  \[x_0 \gets x_1 \gets x_2 \gets \cdots \gets x_k \gets \langle a \rangle\]
  be an augmenting path.
  Define $T$ to be the partial coloring obtained from $S$ by simultaneously recoloring each $x_i$ to the color of $x_{i+1}$ for $i < k$, and recoloring $x_k$ to color $a$.
  Then $T$ is proper.
\end{lemma}
\begin{proof}
  By induction on $k$.  The case $k = 0$ is trivial.

  Let
  \[x_0 \gets x_1 \gets x_2 \gets \cdots \gets x_{k+1} \gets \langle a \rangle\]
  be an augmenting path for $S$.
  Suppose $x_{k+1}$ has color $b$ in $S$.
  Let $U$ be the proper partial coloring obtained from $S$ by recoloring $x_{k+1}$ to color $a$.
  We claim that
  \[x_0 \gets x_1 \gets x_2 \gets \cdots \gets x_k \gets \langle b \rangle\]
  is an augmenting path in $U$:
  \begin{itemize}
  \item $x_k \gets \langle b \rangle$ in $U$:

    Follows from $x_k \gets x_{k+1}$ in $S$.
  \item $x_i \gets x_{i+1}$ in $U$ for $i < k$:

    We can assume $x_{i+1}$ has color $a$, since otherwise this follows from $x_i \gets x_{i+1}$ in $S$.
    We must show $U_a' \in \I_a$, where
    \[U_a' = U_a - x_{i+1} + x_i = S_a - x_{i+1} + x_i + x_k.\]
    We know the following:
    \[
    S_a + x_k &\in \I_a & \text{since $x_k \gets \langle a \rangle$ in $S$} \\
    S_a - x_{i+1} + x_i &\in \I_a & \text{since $x_i \gets x_{i+1}$ in $S$} \\
    S_a + x_i &\not\in \I_a & \text{since $x_i \not\gets \langle a \rangle$ in $S$} \\
    \]
    Since $|S_a + x_k| > |S_a - x_{i+1} + x_i|$
    we must be able to augment $S_a - x_{i+1} + x_i$ by either $x_{i+1}$ or $x_k$.
    The former is impossible so it must be the latter, which shows $U_a' \in \I_a$.
  \item $x_i \not\gets \langle b \rangle$ in $U$ for $i < k$:
  
    Follows from $x_i \not\gets x_{k+1}$ in $S$.
  \item $x_i \not\gets x_j$ in $U$ for $j > i+1$:

    Suppose for contradiction that $x_i \gets x_j$ in $U$.
    We can assume that $x_j$ has color $b$
    since otherwise this contradicts $x_i \not\gets x_j$ in $S$.

    We know the following:
    \[
    S_b &\in \I_b & \text{by assumption} \\
    S_b - x_{k+1} - x_j + x_i &\in \I_b & \text{since $x_i \gets x_j$ in $U$} \\
    S_b - x_j + x_i &\not\in \I_b & \text{since $x_i \not\gets x_j$ in $S$} \\
    S_b - x_{k+1} + x_i &\not\in \I_b & \text{since $x_i \not\gets x_{k+1}$ in $S$} \\
    \]
    Since $|S_b| > |S_b - x_{k+1} - x_j + x_i|$
    we must be able to augment $S_b - x_{k+1} - x_j + x_i$ by either $x_j$ or $x_{k+1}$, but this contradicts the facts.
  \end{itemize}
  Thus
  \[x_0 \gets x_1 \gets x_2 \gets \cdots \gets x_k \gets \langle b \rangle\]
  is an augmenting path in $U$, so $T$ is proper by the inductive hypothesis.
\end{proof}

Now we can describe the full matroid partitioning algorithm.

\begin{theorem}\label{thm:matroid partitioning}
  Let $\I_1, \dots, \I_n$ be matroid independence systems over the ground set $E$ with rank functions $r_1, \dots, r_n$.
  There is a polynomial-time algorithm which finds either a proper coloring,
  or a set $Y \subset E$ such that $|Y| > \sum_i r_i(Y)$.
\end{theorem}

A set $Y$ with $|Y| > \sum_i r_i(Y)$ is a certificate proving that no proper coloring exists, since if $S$ is a proper coloring then
\[|Y| = \sum_i |Y \cap S_i| = \sum_i r_i(Y \cap S_i) \le \sum_i r_i(Y)\]
for any set $Y$.

\begin{proof}[Proof of \cref{thm:matroid partitioning}.]
  As described earlier, the algorithm starts by finding a proper partial coloring $S$ with no augmenting path; the correctness of this part follows from \label{lem:augmenting path}.
  Having found such an $S$, if $S$ is a proper coloring of $E$, then the algorithm outputs $S$.
  Otherwise, there exists at least one uncolored element.
  In this case the algorithm outputs $Y$,
  where $Y$ is the set of elements which can reach an uncolored element via directed paths.
  We must show that $|Y| > \sum_i r_i(Y)$.

  We claim that $r_i(Y) = |Y \cap S_i|$ for each $i$.
  We know $r_i(Y) \ge |Y \cap S_i|$ since $S_i \in \I_i$,
  so suppose for contradiction that $r_i(Y) > |Y \cap S_i|$; this means
  there is an element $y \in Y$ such that $Y \cap S_i + y \in \I_i$.
  We know:
  \[
  S_i &\in \I_i \\
  S_i + y &\not\in \I_i &\text{since $y \not\gets \langle i \rangle$} \\
  \]
  so there is a unique circuit $C \in S_i + y$.
  We cannot have $C \subset Y$ since $Y \cap S_i + y$ is independent,
  so $C$ contains some element $z \in S_i \setminus Y$,
  which means $S_i - z + y \in \I_i$.
  But this is just $y \gets z$, which contradicts $z \notin Y$.
  So the claim is proved.

  Finally we compute
  \[
  \sum_i r_i(Y)
  &= \sum_i |Y \cap S_i| \\
  &= \left|Y \cap \bigcup_i S_i\right| &\text{since the $S_i$ are disjoint} \\
  &< |Y| &\text{since there exists an uncolored element}
  \]
\end{proof}

\section*{Shannon switching game}

\begin{definition}
  Given a matroid $M$ on $E$ and a subset $A \subset E$, the \defn{submatroid} $M|_A$ is a matroid on $A$ where $\I_{M|_A} = \{I \subset A \mid I \in \I_M\}$.
\end{definition}

\begin{definition}
  Given a matroid $M$ on $E$ and a subset $B \subset E$, the \defn{quotient matroid}\footnote{Edmonds uses the notation $M \times (E - B)$ which I feel is confusing.} $M/B$ is a matroid on $E - B$ where $\I_{M/B} = \{I \subset E - B \mid I + B \in \I_M\}$.  Equivalently $\Cc_{M/B} = \{C \cap (E - B) \mid C \in \Cc_M\}$.
\end{definition}

For graphic matroids, the submatroid corresponds to deleting the edges of $A$, and the quotient corresponds to contracting the edges in $B$.  Submatroids and quotients commute with themselves and each other.

\begin{definition}
  The \defn{Shannon switching game} is defined by a matroid $M$ and a subset $F \subset E$.
  Two players, \emph{Cut} and \emph{Short} take turns claiming elements of $E - F$ for themselves, with Cut moving first, until all elements of $E - F$ have been claimed.
  Let $B$ be the set of elements claimed by Short.
  We say that Short wins if $B$ spans $F$; otherwise Cut wins.
\end{definition}

% todo: shannon switching game, duality, matroid packing/covering/intersection connections, TRVB application

\end{document}
