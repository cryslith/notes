 \documentclass[11pt,a4paper]{article}
\usepackage[margin=1in]{geometry}

\usepackage[utf8]{inputenc}
\usepackage{amsmath,amsfonts,amssymb,amsthm}
\usepackage{latexsym}
\usepackage{gensymb}
\usepackage{mathtools}
\usepackage[margin=1in]{geometry}
\usepackage[shortlabels]{enumitem}

\usepackage{libertine}
\usepackage[libertine,vvarbb]{newtxmath}
\usepackage[scaled=0.96]{zi4}
\DeclareMathAlphabet{\mathcal}{OMS}{cmsy}{m}{n}

\usepackage{xcolor}
\usepackage[colorlinks, linkcolor=purple, citecolor=blue!75!black, urlcolor=purple]{hyperref}

\usepackage{tcolorbox}

\usepackage[noabbrev,capitalise]{cleveref}

\newtheorem{theorem}{Theorem}
\newtheorem{lemma}[theorem]{Lemma}
\newtheorem{proposition}[theorem]{Proposition}
\newtheorem{corollary}[theorem]{Corollary}
\newtheorem{definition}[theorem]{Definition}

\newcommand*{\defn}[1]{\textbf{#1}}
\newcommand*{\N}[0]{\mathbb{N}}
\newcommand*{\R}[0]{\mathbb{R}}
\newcommand*{\E}[0]{\mathbf{E}}
\DeclareMathOperator*{\Var}{\mathrm{Var}}
\DeclareMathOperator*{\Cov}{\mathrm{Cov}}
\DeclarePairedDelimiter{\norm}{\lVert}{\rVert}
\newcommand*{\Normal}[0]{\mathcal{N}}
\newcommand*{\Exp}[0]{\text{Exp}}
\newcommand*{\Fl}[0]{\mathcal{F}}
\newcommand*{\tp}[0]{\mathsf{T}}

\def\[#1\]{\begin{align*}#1\end{align*}}
\newcommand*{\pheq}{\mathrel{\phantom{=}}} % invisible = for align mode
\newcommand*{\mat}[1]{\begin{bmatrix}#1\end{bmatrix}}
\renewcommand*{\iff}[0]{\leftrightarrow}

\title{Concentration Bound Tricks}
\author{Lily Chung}
\date{}

\begin{document}
\maketitle

Here I note some simple tricks for applying Chernoff-Hoeffding type concentration inequalities, and collect some useful facts for applications.

\section{High Probability Bounds}

\begin{definition}
  Let \(A_{n, c}\) be an event depending on \(n\) and a value \(c\).
  We say \(A\) holds \defn{with high probability} if
  for any choice of \(c\),
  \(Pr[\overline{A_{n, c}}] \in O(n^{-c})\).\footnote{This notion of ``w.h.p.'' in algorithms differs from other fields which use it to mean that $\Pr[\overline{A_n}] \in o(1)$.}
\end{definition}

In particular we might have a random variable \(X(n)\) depending on \(n\).
If we say, ``\(X\) is \(O(f)\) with high probability'',
we mean that for any constant \(c\), there exist constants \(a, b, N\) such that
\(\Pr[X(n) > af(n)] < bn^{-c}\) for all \(n > N\).
In other words, the constants \(a, b, N\) hidden by the big-O notation are allowed to depend on \(c\).

In particular, suppose \(p\) is a polynomial and we have \(p(n)\) random variables \(X_i\),
such that each \(X_i\) is \(O(f)\) with high probability.
Then by a union bound, \(\max_i X_i\) is \(O(f)\) with high probability.

\section{Chernoff-Hoeffding Bounds}

We will use the following versions of the Chernoff-Hoeffding bounds:

\begin{theorem}[Hoeffding]
  \label{thm:hoeffding}
  Let $X = \sum_{i=1}^k X_i$ where the $X_i$ are independent and $X_i \in [0, 1]$.
  Then \[\Pr[X - \E X \ge t] &\le \exp(-2t^2/k) & t \ge 0\]
\end{theorem}

\begin{theorem}[Chernoff]
  \label{thm:chernoff}
  Let \(X = \sum_i X_i\) where the \(X_i\) are independent and $X_i \in [0, 1]$, and let \(\mu = \E X\).
  Then \[
  \Pr[X \ge (1 + \delta)\mu] &\le \exp\left(-\frac{\delta^2}{2+\delta}\mu\right) & \delta \ge 0 \\
  \Pr[X \le (1 - \delta)\mu] &\le \exp\left(-\frac{\delta^2}{2}\mu\right) & \delta \in (0, 1) \\
  \]
\end{theorem}

Note that the \(X_i\) need not be identically distributed, nor are they required to take only the values \(\{0, 1\}\). % todo prove this via convexity?
One elementary observation is that even without knowing $\mu$ exactly, we can still apply the Chernoff bound using an upper or lower bound on $\mu$:

\begin{corollary}
  \label{cor:chernoff-ineq}
  Let \(X = \sum_i X_i\) such that the \(X_i\) are independent and $X_i \in [0, 1]$,
  and suppose $\ell \le \E X \le u$.
  Then \[
  \Pr[X \ge (1 + \delta)u] &\le \exp\left(-\frac{\delta^2}{2+\delta}u\right) & \delta \ge 0 \\
  \Pr[X \le (1 - \delta)\ell] &\le \exp\left(-\frac{\delta^2}{2}\ell\right) & \delta \in (0, 1) \\
  \]
\end{corollary}
\begin{proof}
  The lower bound in terms of $\ell$ is trivial since all the inequalities are in the correct direction,
  but the upper bound is a bit more difficult because $\exp(-u) \le \exp(-\E X)$.
  To prove it, let \(Y = X + u - \E X\), so that $\E Y = u$ and $X \le Y$.
  Then \(Y\) can be expressed as a sum of independent $[0, 1]$ random variables,
  and applying \cref{thm:chernoff} to \(Y\) gives the result.
\end{proof}

%% \Pr[X \ge 3t] \le \exp(-t).\]  
Using this we can obtain the following asymptotic results:

\begin{tcolorbox}
  \begin{lemma}
    Let \(X = \sum_{i=1}^{k(n)} X_i\) where the \(X_i\) are independent and \(X \in [0, 1]\).
    \begin{itemize}
    \item With high probability $X = \E X \pm O(\sqrt{k(n) \log n})$.
    \item If \(\E X \in O(\log n)\), then with high probability \(X \in O(\log n)\).
    \item If \(\E X \in \Omega(\log n)\), then with high probability \(X \in O(\E X)\).
    \item If \(\E X \in \omega(\log n)\), then with high probability \(X \in \Theta(\E X)\).
    \end{itemize}
  \end{lemma}
\end{tcolorbox}
\begin{proof}\hfill
  \begin{itemize}
  \item
    This follows directly from \cref{thm:hoeffding}.
  \item
    Pick $c$ large enough so that eventually \(\E X \le c \log n\).
    Then we have
    \[\Pr[X \ge 3c \log n] \le \exp(-c\log n) = n^{-c}\]
    by \cref{cor:chernoff-ineq}.
  \item
    Pick any \(c\),
    and let \(d \ge 1\) be a constant large enough so that eventually \(d \E X \ge c \log n\).
    Then \[
    \Pr[X \ge 3d \E X] \le \exp(-d \E X) \le \exp(-c\log n) = n^{-c} \\
    \]
    by \cref{cor:chernoff-ineq}.
  \item
    Pick any $c$; then eventually $\E X \ge 8c \log n$,
    so
    \[
    \Pr[X \le \E X / 2] \le \exp(-\E X / 8) \le \exp(-c\log n) = n^{-c}
    \]
    by \cref{thm:chernoff}.
  \end{itemize}
\end{proof}

% todo amplifying distinguisher (biased coin), mean estimation by sampling

\section{Bounded Independence}
% todo revise section

The Chernoff bound requires that the variables \(X_i\) are fully independent.
In some contexts it can be useful to reduce the amount of independence, and thus the number of random bits needed to generate the variables.
It turns out we still have useful concentration bounds even when the variables are only \(k\)-wise independent.

\begin{theorem}[Theorem 5(IIa) of \cite{limited-chernoff}]
  \label{thm:limited-chernoff}
  Let \(X = \sum_i X_i\) such that the \(X_i\) are \(k\)-wise independent and confined to the range \([0, 1]\) with \(\mu = \E X\).
  Let \(\delta \ge 1\) and suppose \(k \le \lfloor \delta \mu e^{-1/3} \rfloor.\)
  Then \[\Pr[X \ge (1 + \delta)\mu] \le \exp(-\lfloor k/2 \rfloor).\]
\end{theorem}

Again we'll put this into an easier form to apply:

\begin{tcolorbox}
  \begin{corollary}
    \label{cor:limited-chernoff-simple}
    Let \(X = \sum_i X_i\) such that the \(X_i\) are \(k\)-wise independent and confined to the range \([0, 1]\) with \(\mu \ge \E X\).
    Suppose \(k \le \mu.\)
    Then \[\Pr[X \ge 4\mu] \le \exp(-\lfloor k/2 \rfloor).\]
  \end{corollary}
\end{tcolorbox}
\begin{proof}
  Without loss of generality assume \(\mu \ge 2\) since otherwise the statement is trivial.
  Let \(Y = X + \mu - \E X\) as before; then \(Y\) is a sum of \(k\)-wise independent \([0, 1]\) random variables with \(\mu = E[Y]\).
  Set \(\delta = 3\) so that \(k \le \mu \le \lfloor 2\mu \rfloor \le \lfloor \delta \mu e^{-1/3} \rfloor\).
  Applying Theorem~\ref{thm:limited-chernoff} to \(Y\), we obtain
  \(\Pr[X \ge 4\mu] \le \Pr[Y \ge 4\mu] \le \exp(-\lfloor k/2 \rfloor).\)
\end{proof}

We can now show similar results to before but with only \(\Theta(\log n)\) independent bits.

\begin{tcolorbox}
  \begin{corollary}
    Let \(X = \sum_i X_i\) such that the \(X_i\) are \(k\)-wise independent and confined to the range \([0, 1]\), where \(k = 2\lceil c\log n \rceil\).
    \begin{itemize}
    \item If \(\E X \in O(\log n)\), then  \(X \in O(\log n)\) with probability \(1 - n^{-c}\).
    \item If \(\E X \in \Omega(\log n)\), then \(X \in O(\E X)\) with probability \(1 - n^{-c}\).
    \end{itemize}
  \end{corollary}
\end{tcolorbox}
\begin{proof}\hfill
  \begin{itemize}
  \item
    Let \(d \ge k\) be a constant large enough so that eventually \(\E X \le d \log n\).
    Apply Corollary~\ref{cor:limited-chernoff-simple} with \(\mu = d \log n\); 
    then eventually \(\Pr[X \ge 4d \log n] \le \exp(-\lfloor k/2 \rfloor) \le n^{-c}\).
  \item
    Let \(d\) be a constant large enough so that eventually \(d \E X \ge k\).
    Apply Corollary~\ref{cor:limited-chernoff-simple} with \(\mu = d\E X\);
    then eventually \(\Pr[X \ge 4d\E X] \le \exp(-\lfloor k/2 \rfloor) \le n^{-c}\).
  \end{itemize}
\end{proof}

\end{document}
