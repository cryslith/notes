\documentclass[11pt,a4paper]{article}
\usepackage[margin=1in]{geometry}

\usepackage[utf8]{inputenc}
\usepackage{amsmath,amsfonts,amssymb,amsthm}
\usepackage{latexsym}
\usepackage{mathtools}
\usepackage[margin=1in]{geometry}
\usepackage[shortlabels]{enumitem}
\usepackage{comment}

\usepackage{libertine}
\usepackage[libertine,vvarbb]{newtxmath}
\usepackage[scaled=0.96]{zi4}
\DeclareMathAlphabet{\mathcal}{OMS}{cmsy}{m}{n}

\newtheorem{theorem}{Theorem}
\newtheorem{lemma}[theorem]{Lemma}
\newtheorem{corollary}[theorem]{Corollary}
\newtheorem{definition}[theorem]{Definition}
\newtheorem{question}{Question}
\newtheorem{claim}[theorem]{Claim}

\newcommand*{\defn}[1]{\textbf{#1}}
\newcommand*{\N}[0]{\mathbb{N}}
\newcommand*{\R}[0]{\mathbb{R}}

\def\[#1\]{\begin{align*}#1\end{align*}}
\newcommand*{\pheq}{\mathrel{\phantom{=}}} % invisible = for align mode
\newcommand*{\mat}[1]{\begin{bmatrix}#1\end{bmatrix}}
\renewcommand*{\iff}[0]{\leftrightarrow}

\DeclareRobustCommand{\stirling}{\genfrac\{\}{0pt}{}}

\title{Generating Function Cheatsheet}

\author{
  Lily Chung
}
\date{}

\begin{document}
\maketitle

\begin{comment}
  \section*{Unlabeled Prefabs}

  For this section write $A(x) = \sum_{n \ge 0} a_n x^n$.  There are $f_n$ possible prefabs of size $n$.

  An $(A + B)$-prefab is either an $A$-prefab or a $B$-prefab.
  \[cA(x) + dB(x) &= \sum_{n \ge 0} (ca_n + db_n)x^n\]

  A $(\prod_{i \in [k]} A_i)$-prefab of size $n$ consists of an $A_i$-prefab for each $i$, with total size $n$.
  The number $h_n$ of such prefabs satisfies
  \[
  h_n &= \sum_{\substack{v \in \N^k \\ \sum_i v_i = n}} \prod_{i \in [k]} a_{v_i} \\
  \prod_{i \in [k]} A_i(x) &= \sum_{n \ge 0} h_n x^n
  \]

  Assume $B(0) = 0$.  An $(A \circ B)$-prefab of size $n$ consists of a $k$-part composition of $n$, an $A$-prefab of size $k$, and a $B$-prefab of each size in the composition.  The number $r_n$ of such prefabs satisfies
  \[
  r_n &= \sum_k \sum_{\substack{v \in \N_+^k \\ \sum_i v_i = n}} a_k \prod_{i \in [k]} b_{v_i}
  \]
\end{comment}

\section*{Labeled Structures}

For this section write $F(x) = \sum_{n \ge 0} \frac{f_n}{n!}$.
There are $f_n$ possible $F$-structures on a set of size $n$.

An $(F + G)$-structure is either an $F$-structure or a $G$-structure.
\[
cF(x) + dG(x) &= \sum_{n \ge 0} (cf_n + dg_n)\frac{x^n}{n!}
\]

A $(\prod_{i \in [k]} F_i)$-structure on $[n]$ is a map $p : [n] \to [k]$, together with an $F_i$-structure on $p^{-1}(i)$ for each $i \in [k]$.  The number $h_n$ of such structures satisfies
\[h_n &= \sum_{\substack{v \in \N^k \\ \sum_i v_i = n}} \binom{n}{v} \prod_{i \in [k]} f_{v_i} \\
\prod_{i \in [k]} F_i(x) &= \sum_{n \ge 0} h_n \frac{x^n}{n!}
\]

A $F'$-structure on $[n]$ is an $F$-structure on $[n+1]$.
\[
F'(x) &= \sum_{n \ge 0} f_{n+1} \frac{x^n}{n!}
\]

Assume $G(0) = 0$.  A $(F \circ G)$-structure on $[n]$ is a partition of $[n]$ into any number of nonempty parts, a $F$-structure on the set of parts, and a $G$-structure on each part.  The number $r_n$ of such structures satisfies

\[
r_n &= \sum_k \sum_{P \in \stirling{[n]}{k}} f_k \prod_{i \in [k]} g_{|P_i|} \\
F(G(x)) &= \sum_{n \ge 0} r_n \frac{x^n}{n!}
\]

If we write $r_{n, k}$ for the number of such structures with exactly $k$ parts, so $r_n = \sum_k r_{n, k}$, then
\[
F(yG(x)) &= \sum_{\substack{n \ge 0 \\ k \ge 0}} r_{n, k} \frac{x^n}{n!}y^k
\]

\end{document}
