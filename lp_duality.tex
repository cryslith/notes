\documentclass{article}
\usepackage[utf8]{inputenc}
\usepackage{amsmath,amsfonts,amssymb}
\usepackage{amsthm}
\usepackage{mathtools}
\usepackage{tcolorbox}
\usepackage[margin=1in]{geometry}

\newtheorem{theorem}{Theorem}
\newtheorem{lemma}[theorem]{Lemma}
\newtheorem{proposition}[theorem]{Proposition}
\newtheorem{corollary}[theorem]{Corollary}
\newtheorem{definition}[theorem]{Definition}
\newtheorem{question}[theorem]{Question}
\newtheorem{claim}[theorem]{Claim}

\newcommand*{\defn}[1]{\emph{#1}}
\newcommand*{\mat}[1]{\begin{bmatrix}#1\end{bmatrix}}
\newcommand*{\R}[0]{\mathbb{R}}

\title{LP Duality Cheatsheet}
\author{Lily Chung}
\date{}

\begin{document}
\maketitle

\begin{tcolorbox}
  \[
  \begin{gathered}
    \max\mat{c_1 \\ c_2 \\ c_3}^T \mat{x_1 \\ x_2 \\ x_3} \\
    \begin{aligned}
      x_1 &\ge 0 \\
      x_2 &\le 0 \\
      x_3 &\in \R \\
    \end{aligned}\quad
    Ax \in \mat{\le b_\alpha \\ \ge b_\beta \\ = b_\gamma} \\
  \end{gathered}
  \qquad\text{is dual to}\qquad
  \begin{gathered}
    \min\mat{b_\alpha \\ b_\beta \\ b_\gamma}^T \mat{y_\alpha \\ y_\beta \\ y_\gamma} \\
    \begin{aligned}
      y_\alpha \ge 0 \\
      y_\beta \le 0 \\
      y_\gamma \in \R \\
    \end{aligned}\quad
    A^Ty \in \mat{\ge c_1 \\ \le c_2 \\ = c_3} \\
  \end{gathered}
  \]
\end{tcolorbox}
\begin{theorem}[Strong duality]
  One of the following holds:
  \begin{itemize}
  \item Both the primal and dual are feasible and have the same optimum value.
  \item One is unbounded and the other is infeasible.
  \item Both are infeasible.
  \end{itemize}
\end{theorem}

To dualize a linear program:
\begin{itemize}
\item Create a dual variable for every primal constraint.
  The sign of the variable depends\footnote{In some cases there is a sign flip, depending on whether the primal was maximizing or minimizing.  See the boxed equation.} on the direction of the constraint.
  The coefficient of the variable in the objective is the right-hand side of the constraint.
\item For every primal variable \(x\), create a dual constraint in the following way.
  Look at all the primal constraints containing \(x\); for each such constraint multiply the dual variable corresponding to that constraint by the coefficient of \(x\).
  The left-hand side of the new dual constraint is the sum of these terms.
  The right-hand side of the constraint is the coefficient of \(x\) in the objective.
  The direction of the new constraint depends\footnotemark[\value{footnote}]{} on the sign of \(x\).
\end{itemize}

\end{document}
