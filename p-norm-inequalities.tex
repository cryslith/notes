\documentclass{article}
\usepackage[utf8]{inputenc}
\usepackage[margin=1in]{geometry}
\input{liatex/preamble}

\title{$p$-norm inequalities}
\author{Lily Chung}
\date{}

\begin{document}
\maketitle

Let $f : S \to \R_{\ge 0}$ be a nonnegative measurable function on a measure space $(S, \Lambda, \mu)$.
Define \[\norm{f}_p = \begin{cases}
  \left(\int f^p\,d\mu\right)^{1/p} & p \in \R - \{0\} \\
  \exp \left(\int \log f\,d\mu\right) & p = 0 \\
  \essinf f & p = -\infty \\
  \esssup f & p = \infty \\
  \end{cases}
\]
Note that this is not actually a norm unless $p \ge 1$.

\section*{Power Means}

For this section we suppose $\mu$ is a probability measure.

\begin{theorem}[Power Mean Inequality]\label{thm:power mean}
  Let $X$ be a nonnegative random variable.
  If $p \le q$ then $\norm{X}_p \le \norm{X}_q$ whenever both exist.
\end{theorem}
\begin{proof}
  The cases where $p$ or $q$ are infinite are easily checked.

  Suppose $0 = p < q$.  Since $\log$ is concave we have by Jensen's inequality
  \[
  \log \E[X^q] &\ge \E[\log X^q] \\
  &= q \E[\log X] \\
  \E[X^q]^{1/q} &\ge \exp \E[\log X] \\
  \]
  The case where $p < q = 0$ is analogous.

  Now suppose $p < q$ with $p \ne 0, q > 0$.  Since $u \mapsto u^{q/p}$ is convex we have
  \[
  \E[X^p]^{q/p} &\le \E[X^q] \\
  \E[X^p]^{1/p} &\le \E[X^q]^{1/q} \\
  \]
  The case where $p < q < 0$ is analogous.
\end{proof}

For instance this generalizes the AM-GM inequality since $\norm{\cdot}_0$ is the geometric mean and $\norm{\cdot}_1$ is the arithmetic mean.

\section*{H\"older's Inequality}

\begin{theorem}[H\"older's Inequality]
  Let $f, g : S \to \R$ be nonnegative measurable functions,
  and let $p, q \in [-\infty, \infty]$ such that $\frac1p + \frac1q = 1$ or $p=q=0$.
  Then:
  \begin{itemize}
  \item if $p, q \in [1, \infty]$ then $\norm{fg}_1 \le \norm{f}_p \norm{g}_q$
  \item if $p, q \in [-\infty, 1]$ then $\norm{fg}_1 \ge \norm{f}_p \norm{g}_q$
  \item if $p = q = 0$ then $\norm{fg}_1 = \norm{f}_p \norm{g}_q$
  \end{itemize}
  provided that the integrals exist.
\end{theorem}
\begin{proof}
  The $-\infty, 0, \infty$ cases are easily checked.
  Define $\nu$ to be a probability measure on $S$ via $d\nu = \frac{g^q}{\int g^q\,d\mu} d\mu$,
  and let $X = fg^{1-q}$.  (We can assume $\int g^q\,d\mu \in (0, \infty)$ since otherwise the result is trivial.)
  Then we have
  \[\norm{fg}_1 &= \left(\int g^q\,d\mu\right)\left(\int f g^{1-q} \frac{g^q}{\int g^q\,d\mu}\,d\mu\right) \\
  &= \left(\int g^q\,d\mu\right)\E_\nu[X] \\
  &\le \left(\int g^q\,d\mu\right)\E_\nu[X^p]^{1/p} &\text{by \cref{thm:power mean}}\\
  &= \left(\int g^q\,d\mu\right)\left(\int f^p g^{p(1-q)}\frac{g^q}{\int g^q\,d\mu}\,d\mu \right)^{1/p} \\
  &= \left(\int g^q\,d\mu\right)^{1 - 1/p}\left(\int f^p g^{p + q - pq}\,d\mu \right)^{1/p} \\
  &= \norm{f}_p \norm{g}_q \\
  \]
  where the sign of the inequality is flipped if $p, q < 1$.
\end{proof}

\end{document}
