\documentclass[11pt,a4paper]{article}
\usepackage[margin=1in]{geometry}

\usepackage[utf8]{inputenc}
\usepackage{amsmath,amsfonts,amssymb,amsthm}
\usepackage{latexsym}
\usepackage{mathtools}
\usepackage[margin=1in]{geometry}
\usepackage[shortlabels]{enumitem}

\usepackage{xcolor}
\usepackage[colorlinks, linkcolor=purple, citecolor=blue!75!black, urlcolor=purple]{hyperref}
\usepackage[noabbrev,capitalise]{cleveref}

\usepackage{libertine}
\usepackage[libertine,vvarbb]{newtxmath}
\usepackage[scaled=0.96]{zi4}
\DeclareMathAlphabet{\mathcal}{OMS}{cmsy}{m}{n}

\newtheorem{theorem}{Theorem}
\newtheorem{lemma}[theorem]{Lemma}
\newtheorem{corollary}[theorem]{Corollary}
\newtheorem{definition}[theorem]{Definition}
\newtheorem{question}{Question}
\newtheorem{claim}[theorem]{Claim}

\newcommand*{\defn}[1]{\textbf{#1}}
\newcommand*{\N}[0]{\mathbb{N}}
\newcommand*{\R}[0]{\mathbb{R}}
\newcommand*{\E}[0]{\mathbf{E}}
\DeclareMathOperator*{\Var}{\mathrm{Var}}
\DeclareMathOperator*{\Cov}{\mathrm{Cov}}
\DeclarePairedDelimiter{\norm}{\lVert}{\rVert}
\newcommand*{\tp}[0]{\mathsf{T}}

\def\[#1\]{\begin{align*}#1\end{align*}}
\newcommand*{\pheq}{\mathrel{\phantom{=}}} % invisible = for align mode
\newcommand*{\mat}[1]{\begin{bmatrix}#1\end{bmatrix}}
\renewcommand*{\iff}[0]{\leftrightarrow}

\newcommand*{\Lip}[0]{\mathsf{Lip}}
\newcommand*{\Mon}[0]{\mathsf{Mon}}

\title{Distance to Lipschitz}

\author{
  Lily Chung, Jane Lange
}
\date{}

\begin{document}
\maketitle

In this note we show that the $\ell_1$ distance from a real-valued function $f$ to the closest Lipschitz function is the minimum weight of a fractional vertex cover of the violation graph.
The proof is simpler than the existing proof of this fact in the literature.
We also cover the analogous statements for monotonicity and for $\ell_0$ distance.

\begin{definition}
  Let $X$ be a finite set and $C$ be a class of functions $X \to \R$.
  If $f : X \to \R$ then we define the $\ell_p$ distance from $f$ to $C$ as
  \[
  d_p(f, C) &= \inf_{g \in C} \left(\sum_{x \in X} |f(x) - g(x)|^p\right)^{1/p} & p \ge 1 \\
  d_0(f, C) &= \inf_{g \in C} \#\{x \in X : f(x) \ne g(x)\} \\
  \]
\end{definition}

We write $\Lip(X)$ for the class of Lipschitz functions on a metric space $X$, and $\Mon(X)$ for the class of monotone functions on a poset $X$.

\begin{definition}
  Let $X$ be a finite metric space and $f : X \to \R$.
  For any pair of elements $x, y \in X$ the \defn{Lipschitz violation score} induced by $f$
  is \[V_{xy} = |f(x) - f(y)| - d(x, y)\]
  The \defn{Lipschitz violation graph} of $f$ is the weighted graph whose vertices are elements of $X$
  and which has an edge $(x, y)$ whenever $V_{xy} > 0$.
\end{definition}

\begin{definition}
  Let $X$ be a finite metric space and $f : X \to \R$.
  For any pair of elements $x \le y \in X$ the \defn{monotone violation score} induced by $f$
  is \[M_{xy} = f(x) - f(y)\]
  The \defn{monotone violation graph} of $f$ is the weighted graph whose vertices are elements of $X$
  and which has an edge $(x, y)$ whenever $M_{xy} > 0$.
\end{definition}

\begin{definition}
  A \defn{vertex cover} of a graph $G = (V, E)$ is a set $S$ such that $\{u, v\} \cap S \ne \emptyset$ for each edge $(u, v) \in E$.
\end{definition}

\begin{definition}
  A \defn{fractional vertex cover} of a weighted graph $G = (V, E, w)$ is a function $b : V \to \R_{\ge 0}$ such that $b_u + b_v \ge w_{uv}$ for each edge $(u, v) \in E$.
  The \defn{weight} of $b$ is $\sum_{v \in V} b_v$.
\end{definition}

\begin{theorem}
  Let $X$ be a finite metric space and $f : X \to \R$.
  Then $d_1(f, \Lip(X))$ is equal to the minimum weight $W$ of
  a fractional vertex cover of the Lipschitz violation graph of $f$.
\end{theorem}
\begin{proof}
  The direction $d_1(f, \Lip(X)) \ge W$ is easy: for every Lipschitz function $g$
  the fractional vertex cover $x \mapsto |f(x) - g(x)|$
  has weight exactly $\norm{f - g}_1$.
  It is a fractional vertex cover because
  \[
  |f(x) - g(x)| + |f(y) - g(y)|
  &\ge |f(x) - f(y)| - |g(x) - g(y)| \\
  &\ge |f(x) - f(y)| - d(x, y) \\
  &= V_{xy}
  \]

  Now suppose $b$ is a fractional vertex cover of the Lipschitz violation graph of $f$.
  Define
  \[g(x) = \sup_{z \in X} (f(z) - b_z - d(x, z))\]
  We claim $g$ is Lipschitz.  Let $x, y \in X$; without loss of generality $g(x) \le g(y)$.
  Then indeed
  \[
  g(y) - g(x) &= \sup_{z \in X} (f(z) - b_z - d(y, z) - g(x)) \\
  &\le \sup_{z \in X} (f(z) - b_z - d(y, z) - (f(z) - b_z - d(x, z))) \\
  &= \sup_{z \in X} (d(x, z) - d(y, z)) \\
  &\le d(x, y)
  \]

  We also claim that $\norm{f - g}_1$ is at most the weight of $b$;
  it suffices to show that $g(x) \in [f(x) - b_x, f(x) + b_x]$ for each $x$.
  The lower bound $g(x) \ge f(x) - b_x$ is obvious since we can choose $x$ in the supremum.
  As for the upper bound, we need to show $f(z) - b_z - d(x, z) \le f(x) + b_x$ for every $z \in X$.
  In the case $f(z) \le f(x)$ this inequality is trivial,
  and when $f(z) \ge f(x)$ it is exactly the fractional vertex cover property.

  This shows $d_1(f, \Lip(X)) \le W$ and concludes the proof.
\end{proof}

The following results follow essentially the same argument, so we just give the critical part.

\begin{theorem}
  Let $X$ be a finite metric space and $f : X \to \R$.
  Then $d_0(f, \Lip(X))$ is equal to the minimum size of
  a vertex cover of the unweighted Lipschitz violation graph of $f$.
\end{theorem}
\begin{proof}[Proof sketch.]
  Given a vertex cover $S$,
  consider $g(x) = \sup_{z \in X - S}(f(z) - d(x, z))$.
\end{proof}

\begin{theorem}
  Let $X$ be a finite poset and $f : X \to \R$.
  Then $d_1(f, \Mon(X))$ is equal to the minimum weight of
  a fractional vertex cover of the monotone violation graph of $f$.
\end{theorem}
\begin{proof}[Proof sketch.]
  Given a fractional vertex cover $b$,
  consider $g(x) = \sup_{z \le x} (f(z) - b_z)$.
\end{proof}

\begin{theorem}
  Let $X$ be a finite poset and $f : X \to \R$.
  Then $d_0(f, \Mon(X))$ is equal to the minimum weight of
  a vertex cover of the unweighted monotone violation graph of $f$.
\end{theorem}
\begin{proof}[Proof sketch.]
  Given a vertex cover $S$,
  consider $g(x) = \sup\{f(z) : z \in X - S \land z \le x\}$.
\end{proof}

\end{document}
