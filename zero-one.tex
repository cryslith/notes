\documentclass[11pt,a4paper]{article}
\usepackage[utf8]{inputenc}
\usepackage[margin=1in]{geometry}
\input{liatex/preamble.tex}

\title{Kolmogorov's Zero-One Law}

\author{
  Lily Chung
}
\date{}

\begin{document}
\maketitle

This note rigorously proves the zero-one law.

We start by reviewing a basic property of set systems.

\begin{definition}
  Fix a universe $U$, and let $S \subset U$.
  \begin{itemize}
  \item $S$ is a \defn{$\pi$-system} if it is closed under finite intersection.
  \item $S$ is a \defn{$\lambda$-system} if it is closed under complement and countable disjoint union.
  \item $S$ is a \defn{$\sigma$-algebra} if it is closed under complement and countable union.
  \end{itemize}
\end{definition}

We write $\sigma(S)$ for the smallest sigma-algebra containing a set system $S$, and similarly for $\lambda(S)$ and $\pi(S)$.

\begin{theorem}[Dynkin $\pi$-$\lambda$]
  \label{thm:dynkin}
  Let $S$ be a $\pi$-system.  Then $\sigma(S) = \lambda(S)$.
\end{theorem}
\begin{proof}
  It suffices to show that $\lambda(S)$ is closed under intersection,
  since then it is a $\sigma$-algebra.
  Define an undirected graph whose vertices are subsets of $U$,
  and which has an edge $(A, B)$ if $A \cap B \in \lambda(S)$.

  For each $A \in \lambda(S)$ its neighborhood $\Gamma(A)$ is a $\lambda$-system:
  \begin{itemize}
  \item $A \cap \overline{B} = \overline{\overline{A} \dotcup (A \cap B)}$
  \item $A \cap \left(\bigdotcup_i B_i\right) = \bigdotcup_i (A \cap B_i)$
  \end{itemize}
  So by definition of $\lambda(S)$, it follows that any vertex in $\lambda(S)$ which is adjacent to every element of $S$ must in fact be adjacent to every element of $\lambda(S)$.

  Since $S$ is a $\pi$-system it forms a clique in this graph.
  By the above property there is an edge between every element of $S$ and every element of $\lambda(S)$.
  Finally, using the property again we find $\lambda(S)$ is a clique,
  thus it is closed under intersection as desired.
\end{proof}

Now fix some probability space for the rest of the discussion to take place in.

\begin{lemma}
  \label{lem:pi ind}
  Let $\{A_i : i \in I\}$ be an independent collection of $\pi$-systems.
  Then $\{\sigma(A_i) : i \in I\}$ are independent also.
\end{lemma}
\begin{proof}
  It suffices to consider a finite independent collection $\{A_1, \dots, A_n\}$ of $\pi$-systems.
  We start by showing the simpler claim that $\{\sigma(A_1)\} \cup \{A_i : i > 1\}$ is independent.

  Pick an event $E_1 \in \sigma(A_1)$
  and events $E_i \in A_i$ for $i > 1$; we will show that these events are independent.
  Let $L$ be the set of events independent of $\bigwedge_{i=2}^n E_i$.
  Then $L$ is a $\lambda$-system which contains $A_1$, so by Theorem~\ref{thm:dynkin} it contains $\sigma(A_1)$ also.

  We thus have
  \[
  \Pr\left[\bigwedge_{i=1}^n E_i\right]
   = \Pr[E_1]\Pr\left[\bigwedge_{i=2}^n E_i\right]
   = \prod_{i=1}^{n} \Pr[E_i]
  \]
  as desired.

  To prove the lemma, we can use induction on $k$ to show that $\{\sigma(A_i) : i \le k\} \cup \{A_i : i > k\}$ is independent, since the inductive step is just the above argument.
\end{proof}

\begin{lemma}
  \label{lem:partition ind}
  Let $\{A_i : i \in I\}$ be an independent collection of $\pi$-systems,
  and let $\{P_j : j \in J\}$ be a partition of $I$.
  Then $\left\{\sigma(B_j) : j \in J\right\}$ is independent, where $B_j = \bigcup_{i \in P_j} A_i$.
\end{lemma}
\begin{proof}
  By Lemma~\ref{lem:pi ind} it suffices to show that
  $\{\pi(B_j) : j \in J\}$ is independent.
  An element of $\pi(B_j)$ is just a finite intersection
  of elements of $A_i$ for $i \in P_j$,
  so this follows easily.
\end{proof}

\begin{definition}
  Let $\{A_i : i \in \N\}$ be an independent sequence of $\pi$-systems.
  The \defn{tail $\sigma$-algebra} is \[T = \bigcap_{n \in \N} \sigma\left(\bigcup_{i > n} A_i\right)\]
\end{definition}

\begin{theorem}
  Every event in $T$ has probability zero or one.
\end{theorem}
\begin{proof}
  For any finite set $n$ let $B_n = \sigma\left(\bigcup_{i \le n} A_i\right)$ and $C_n = \sigma\left(\bigcup_{i > n} A_i\right)$.
  Fix $n$, then $B_n \Perp C_n$ by \cref{lem:partition ind}, and so $T \Perp B_n$ since $T \subset C_n$.

  Thus $T \Perp \bigcup_{n \in \N} B_n$.  But $\bigcup_{n \in \N} B_n$ is a $\pi$-system, so in fact
  \[T &\Perp \sigma\left(\bigcup_{n \in \N} B_n\right) &\text{\cref{lem:pi ind}}\\
  &= \sigma\left(\bigcup_{n \in \N} A_n\right)\]
  Since $T \subset \sigma\left(\bigcup_{n \in \N} A_n\right)$
  we conclude $T \Perp T$, which is the desired result.
\end{proof}

From the proof we can also see that any event in $\sigma\left(\bigcup_n A_n\right)$ which is independent from $B_n$ for every $n$ has probability zero or one.

\end{document}
