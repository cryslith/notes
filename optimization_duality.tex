\documentclass{article}
\usepackage[utf8]{inputenc}
\usepackage[margin=1in]{geometry}
\usepackage{tcolorbox}
\input{liatex/preamble}

\usepackage[style=alphabetic]{biblatex}
\addbibresource{notes.bib}

\DeclareMathOperator{\cl}{cl}
\DeclareMathOperator{\cvx}{cvx}
\DeclareMathOperator{\aff}{aff}
\DeclareMathOperator{\interior}{int}
\DeclareMathOperator{\relint}{ri}
\DeclareMathOperator{\cvxcl}{\breve{cl}}
\DeclareMathOperator{\ccvcl}{\hat{cl}}
\DeclareMathOperator{\epi}{epi}

\title{Convex optimization notes}
\author{Lily Chung}
\date{}

\begin{document}
\maketitle

Notes accompanying \cite{Bertsekas2009}.
Everything is over $\R^n$.

Recall that the convex closure $\cvxcl f$ of a function $f$ is the convex l.s.c. function whose epigraph is $\cl(\cvx(\epi(f)))$, the convex closure of the epigraph of $f$.  The \defn{concave closure} $\ccvcl f$ is defined as $\ccvcl f = -(\cvxcl(-f))$.

\section*{Convex conjugate}

Let $f : \R^n \to [-\infty, \infty]$.

\[
f^* &: \R^n \to [-\infty, \infty] &\text{convex, lower semicontinuous} \\
f^*(y) &\eqdef \sup_x\{y^\tp x - f(x)\} \\
\marginnote{weak duality} f^{**} &\le f \\
(\cvxcl f)^* &= f^* \\
\marginnote{strong duality} (\cvxcl f) &= f^{**} &\text{if $\cvxcl f$ is proper} \\
\]

\begin{tcolorbox}[title=Geometric intuition]
  Define the function

  \[q(y) &\eqdef \inf_x\{y^\tp x + f(x)\} = -f^*(-y) &\text{concave, upper semicontinuous}\]

  Geometrically $q(y)$ is the highest possible intersection between an underlying hyperplane of $\epi(f)$ with normal $(y, 1)$, and the vertical line through the origin.  The hyperplane itself is the graph of $x \mapsto -y^\tp x + q(y)$, and the double dual is $f^{**}(x) = (-q)^*(-x) = \sup_y\{-y^\tp x + q(y)\}$.
\end{tcolorbox}

\begin{itemize}
\item If $f$ is convex, then properness of any of $f, f^*, f^{**}$ implies properness of all of them.
\item If $\cvxcl f$ is improper because $\cvxcl f = \infty$, then $f = \infty$ also.  In this case $f^* = -\infty$ and strong duality holds.
\item If $\cvxcl f$ is improper because some $(\cvxcl f)(x) = -\infty$, then $f^* = \infty$.  Thus strong duality only holds if $f = -\infty$.
\item Suppose $f$ is convex improper and not identically $\infty$.  Then $f$ equals $-\infty$ on a nonempty convex set $S$, equals $\infty$ outside $\cl(S)$, and may take any value on the relative boundary of $S$.  We can phrase this as:
  \begin{itemize}
  \item Inside $S$, strong duality holds ($f = f^{**} = -\infty$).
  \item Outside $\cl(S)$, strong duality fails ($f^{**} = -\infty < \infty = f$).
  \item On $\cl(S) - S$, strong duality fails and $f$ is not l.s.c.
  \end{itemize}
\end{itemize}

Now suppose that $f$ is convex proper.  We can say the following:
\begin{itemize}
\item If $y \in \dom(f^*)$ then $y^\tp d \le r_{\cl f}(d)$ holds for all directions $d$
  [c.f. Exercise 1.64], though the converse is only true if $\dom(f^*)$ is closed (consider for instance $f(x) = -\ln(x), f^*(y) = \ln(-y) - 1$).
  In particular, $\dom(f^*) \subset R_{\cl f}^*$ [Proposition~4.5.1].
\item Since $f$ is continuous over $\relint(\dom(f))$ [Proposition~1.3.11], it follows that $f$ agrees with $\cvxcl f$ and strong duality holds there.
  Furthermore, in this case the set $Q^*$ of optimal solutions $y$ in the maximization $f^{**}(x) = \sup_y\{-y^\tp x + q(y)\}$ is nonempty [Proposition~4.4.1], and decomposes [Proposition~4.4.2] as\footnote{%
  Here $(\aff(\dom(f)))^\bot$ is the space of vectors orthogonal to all vectors parallel to $\aff(\dom(f))$.
  } $Q^* = (\aff(\dom(f)))^\bot + \tilde Q$ where $\tilde Q$ is nonempty, convex, and compact.
\item For points on the relative boundary, the above may not hold.  Strong duality can fail due to $f$ not being l.s.c; and the set of optimal solutions may be empty (e.g. $f(x) = -\sqrt{1 - x^2}$) or noncompact (e.g. $f$ is an indicator function).
\end{itemize}

\[
(x \mapsto f(ax))^*(y) &= f^*\left(\frac{y}{a}\right) & a \ne 0 \\
(af)^*(y) &= a f^*\left(\frac{y}{a}\right) & a > 0 \\
\sup_x\{y^\tp x + f(x)\} &= (-f)^*(y) \\
\sup_x\{- y^\tp x + f(x)\} &= (-f)^*(-y) \\
\sup_x\{- y^\tp x - f(x)\} &= f^*(-y) \\
\inf_x\{y^\tp x + f(x)\} &= -f^*(-y) \\
\inf_x\{y^\tp x - f(x)\} &= -(-f)^*(-y) \\
\inf_x\{- y^\tp x + f(x)\} &= -f^*(y) \\
\inf_x\{- y^\tp x - f(x)\} &= -(-f)^*(y) \\
\]

\begin{tcolorbox}[title=Min-common / Max-crossing interpretation]
  Let $M$ be a nonempty subset of $\R^{n+1}$.

  \[
  p(u) &\eqdef \inf\{w : (u, w) \in M\} \\
  q(\mu) &\eqdef \inf_{(u, w) \in M}\{\mu^\tp u + w\} &\text{concave, upper semicontinuous} \\
  &= \inf_u \{\mu^\tp u + p(u)\} \\
  &= -p^*(-\mu) \\
  \marginnote{min common} NC &\eqdef \inf\{w : (0, w) \in M\} \\
  &= p(0) \\
  \marginnote{max crossing} XC &\eqdef \sup\{b : \mu^\tp u + w \ge b \quad\forall (u, w) \in M\} \\
  &= \sup_\mu q(\mu) \\
  &= p^{**}(0) \\
  \marginnote{weak duality} XC &\le NC \\
  \]
  \begin{itemize}
  \item
    $p, q, NC, XC$ are unchanged if we replace $M$ by its upward closure $\overline{M} = M + \{(0, w) : w \ge 0\}$ or by $\epi(p)$.
  \item $q$ and $XC$ are unchanged if we replace $M$ by $\cvx(M)$ or $\cl(M)$.
  \end{itemize}
\end{tcolorbox}



\section*{Optimization}

Let $f$ be a function to be minimized, $F(x, u)$ chosen such that $f(x) = F(x, 0)$.

\[
p(u) &\eqdef \inf_x F(x, u) \\
p(0) &= \inf_x F(x, 0) \\
&= \inf_x f(x) \\
q(\mu) &\eqdef -p^*(-\mu) \\
&= \inf_{x, u}\{F(x, u) + \mu^\tp u\} \\
&= -F^*(0, -\mu) \\
p^{**}(0) &= \sup_\mu q(\mu) \\
&= -\inf_\mu F^*(0, \mu) \\
\]

\section*{Minimax}

Let $\phi : X \times Z \to \R$ where $X \subset \R^n, Z \subset \R^n$.

\[
\phi_x(z) &\eqdef \phi(x, z) \\
p(u) &\eqdef \inf_x \sup_z \{\phi(x, z) - u^\tp z\} \\
&= \inf_x (-\phi_x)^*(-u) \\
q(\mu) &\eqdef -p^*(-\mu) \\
&= \inf_u \inf_x \sup_z \{\phi(x, z) - u^\tp z + u^\tp \mu\} \\
&= \inf_x \{-(-\phi_x)^{**}(\mu)\} \\
\sup_z \inf_x \phi(x, z)
\le \sup_\mu \inf_x \{-(-\phi_x)^{**}(\mu)\}
= p^{**}(0)
&\le p(0)
= \inf_x \sup_z \phi(x, z) \\
&= \inf_x \sup_z \{ (\ccvcl \phi_x)(z) \} \\
\]

\begin{itemize}
\item If $-\ccvcl \phi_x$ is proper for each $x$, then we have $-(-\phi_x)^{**} = \ccvcl \phi_x$.
\item If additionally $\phi_x = \ccvcl \phi_x$ for each $x$, then
  the minimax equality $\inf_x \sup_z \phi(x, z) = \sup_z \inf_x \phi(x, z)$ becomes equivalent to the strong duality inequality $p^{**}(0) \le p(0)$.
\end{itemize}

\printbibliography

\end{document}
