\documentclass{article}
\usepackage[utf8]{inputenc}
\usepackage[margin=1in]{geometry}
\usepackage{tcolorbox}
\input{liatex/preamble}

\usepackage[style=alphabetic]{biblatex}
\addbibresource{notes.bib}

\DeclareMathOperator{\cl}{cl}
\DeclareMathOperator{\cvx}{cvx}
\DeclareMathOperator{\aff}{aff}
\DeclareMathOperator{\interior}{int}
\DeclareMathOperator{\relint}{ri}
\DeclareMathOperator{\cvxcl}{\breve{cl}}
\DeclareMathOperator{\ccvcl}{\hat{cl}}
\DeclareMathOperator{\epi}{epi}

\title{Convex optimization notes}
\author{Lily Chung}
\date{}

\begin{document}
\maketitle

Notes accompanying \cite{Bertsekas2009}.
Everything is over $\R^n$.

Recall that the convex closure $\cvxcl f$ of a function $f$ is the convex l.s.c. function whose epigraph is $\cl(\cvx(\epi(f)))$, the convex closure of the epigraph of $f$.  The \defn{concave closure} $\ccvcl f$ is defined as $\ccvcl f = -(\cvxcl(-f))$.

\section*{Convex conjugate}

Let $f : \R^n \to [-\infty, \infty]$.

\[
f^* &: \R^n \to [-\infty, \infty] &\text{convex, lower semicontinuous} \\
f^*(y) &\eqdef \sup_x\{y^\tp x - f(x)\} \\
\marginnote{weak duality} f^{**} &\le f \\
(\cvxcl f)^* &= f^* \\
\marginnote{strong duality} (\cvxcl f) &= f^{**} &\text{if $\cvxcl f$ is proper} \\
\]

\begin{tcolorbox}[title=Geometric intuition]
  Define the function

  \[q(y) &\eqdef \inf_x\{y^\tp x + f(x)\} = -f^*(-y) &\text{concave, upper semicontinuous}\]

  Geometrically $q(y)$ is the highest possible intersection between an underlying hyperplane of $\epi(f)$ with normal $(y, 1)$, and the vertical line through the origin.  The hyperplane itself is the graph of $x \mapsto -y^\tp x + q(y)$, and the double dual is $f^{**}(x) = (-q)^*(-x) = \sup_y\{-y^\tp x + q(y)\}$.
\end{tcolorbox}

\begin{itemize}
\item If $f$ is convex, then properness of any of $f, f^*, f^{**}$ implies properness of all of them.
\item If $\cvxcl f$ is improper because $\cvxcl f = \infty$, then $f = \infty$ also.  In this case $f^* = -\infty$ and strong duality holds.
\item If $\cvxcl f$ is improper because some $(\cvxcl f)(x) = -\infty$, then $f^* = \infty$.  Thus strong duality only holds if $f = -\infty$.
\item Suppose $f$ is convex improper and not identically $\infty$.  Then $f$ equals $-\infty$ on a nonempty convex set $S$, equals $\infty$ outside $\cl(S)$, and may take any value on the relative boundary of $S$.  We can phrase this as:
  \begin{itemize}
  \item Inside $S$, strong duality holds ($f = f^{**} = -\infty$).
  \item Outside $\cl(S)$, strong duality fails ($f^{**} = -\infty < \infty = f$).
  \item On $\cl(S) - S$, strong duality fails and $f$ is not l.s.c.
  \end{itemize}
\end{itemize}

Now suppose that $f$ is convex proper.  We can say the following:
\begin{itemize}
\item If $y \in \dom(f^*)$ then $y^\tp d \le r_{\cl f}(d)$ holds for all directions $d$
  [c.f. Exercise 1.64], though the converse is only true if $\dom(f^*)$ is closed (consider for instance $f(x) = -\ln(x), f^*(y) = \ln(-y) - 1$).
  In particular, $\dom(f^*) \subset (R_{\cl f})^*$ [Proposition~4.5.1].
\item Since $f$ is continuous over $\relint(\dom(f))$ [Proposition~1.3.11], it follows that $f$ agrees with $\cvxcl f$ and strong duality holds there.
  Furthermore, in this case the set $Q^*$ of dual optimal solutions (i.e. optimal $y$ in the maximization $f^{**}(x) = \sup_y\{-y^\tp x + q(y)\}$) is nonempty [Proposition~4.4.1],
  and decomposes [Proposition~4.4.2] as $Q^* = V^\bot + \tilde Q$
  where $V$ is the subspace parallel to $\aff(\dom(f))$,
  and $\tilde Q$ is nonempty, convex, and compact.
\item For points on the relative boundary, the above may not hold.  Strong duality can fail due to $f$ not being l.s.c; and the set of optimal solutions may be empty (e.g. $f(x) = -\sqrt{1 - x^2}$) or noncompact (e.g. $f$ is an indicator function).
\end{itemize}

\begin{tcolorbox}[title=Summary for convex $f$]
  \begin{itemize}
  \item Duality gap ($f^{**}(x) < f(x)$) can arise when:
    \begin{itemize}
    \item $f$ attains the value $-\infty$ somewhere, and $f(x) = \infty$;
    \item or $f$ is not l.s.c. at $x$.
    \end{itemize}
  \item Dual optimal solutions fail to exist when:
    \begin{itemize}
    \item $f(x) = \infty$;
    \item or there is no nonvertical supporting hyperplane at $x$.
    \end{itemize}
  \end{itemize}
\end{tcolorbox}

\[
(x \mapsto f(ax))^*(y) &= f^*\left(\frac{y}{a}\right) & a \ne 0 \\
(af)^*(y) &= a f^*\left(\frac{y}{a}\right) & a > 0 \\
\sup_x\{y^\tp x + f(x)\} &= (-f)^*(y) \\
\sup_x\{- y^\tp x + f(x)\} &= (-f)^*(-y) \\
\sup_x\{- y^\tp x - f(x)\} &= f^*(-y) \\
\inf_x\{y^\tp x + f(x)\} &= -f^*(-y) \\
\inf_x\{y^\tp x - f(x)\} &= -(-f)^*(-y) \\
\inf_x\{- y^\tp x + f(x)\} &= -f^*(y) \\
\inf_x\{- y^\tp x - f(x)\} &= -(-f)^*(y) \\
\]

\begin{tcolorbox}[title=Min-common / Max-crossing interpretation]
  Let $M$ be a nonempty subset of $\R^{n+1}$.

  \[
  p(u) &\eqdef \inf\{w : (u, w) \in M\} \\
  q(\mu) &\eqdef \inf_{(u, w) \in M}\{\mu^\tp u + w\} &\text{concave, upper semicontinuous} \\
  &= \inf_u \{\mu^\tp u + p(u)\} \\
  &= -p^*(-\mu) \\
  \marginnote{min common} NC &\eqdef \inf\{w : (0, w) \in M\} \\
  &= p(0) \\
  \marginnote{max crossing} XC &\eqdef \sup\{b : \mu^\tp u + w \ge b \quad\forall (u, w) \in M\} \\
  &= \sup_\mu q(\mu) \\
  &= p^{**}(0) \\
  \marginnote{weak duality} XC &\le NC \\
  \]
  \begin{itemize}
  \item
    $p, q, NC, XC$ are unchanged if we replace $M$ by its upward closure $\overline{M} = M + \{(0, w) : w \ge 0\}$ or by $\epi(p)$.
  \item $q$ and $XC$ are unchanged if we replace $M$ by $\cvx(M)$ or $\cl(M)$.
  \end{itemize}
\end{tcolorbox}



\section*{Optimization}

Let $f : \R^n \to [-\infty, \infty]$ be a function to be minimized, $F(x, u)$ chosen such that $f(x) = F(x, 0)$.

\[
p(u) &\eqdef \inf_x F(x, u) \\
p(0) &= \inf_x F(x, 0) \\
&= \inf_x f(x) \\
q(\mu) &\eqdef -p^*(-\mu) \\
&= \inf_{x, u}\{F(x, u) + \mu^\tp u\} \\
&= -F^*(0, -\mu) \\
p^{**}(0) &= \sup_\mu q(\mu) \\
&= -\inf_\mu F^*(0, \mu) \\
\]

\subsection*{Lagrangian}

Minimizing $f : X \to \R$ over $\{x \in X : g(x) \le 0\}$.

\[
p(u) &\eqdef \inf_{x \in X : g(x) \le u} f(x) \\
q(\mu) &= \inf_u\{p(u) + \mu^\tp u\} \\
&=
\begin{cases}
  \inf_{x \in X} L(x, \mu) & \mu \ge 0 \\
  -\infty & \text{otherwise}
\end{cases} \\
\marginnote{Lagrangian} L(x, \mu) &\eqdef f(x) + \mu^\tp g(x) \\
\]

\begin{tcolorbox}[title=Optimality conditions (Proposition~5.3.2)]
  Consider the nonlinear program
  \[
  z = \inf\{f(x) : x \in X, g(x) \le 0\}
  \]
  where $X \subset \R^n$ is any set and $f, g_i : X \to \R$ are any functions.

  The Lagrangian is
  \[
  L(x, \mu) &= f(x) + \mu^\tp g(x).
  \]

  The following are equivalent:
  \begin{itemize}
  \item $x^*$ is primal optimal, $\mu^*$ is dual optimal,
    and strong duality holds, meaning $z = \sup_{\mu \ge 0} \inf_{x \in X} L(x, \mu)$.
  \item The following conditions all hold:
    \[
    \marginnote{primal feasibility} x^* &\in \{x \in X : g(x) \le 0\} \\
    \marginnote{dual feasibility} \mu^* &\ge 0 \\
    L(x^*, \mu^*) &= \min_{x \in X} L(x, \mu^*) \\
    \marginnote{complementary slackness} {\mu^*}^\tp g(x^*) &= 0.
    \]
  \end{itemize}
\end{tcolorbox}

Note that complementary slackness is equivalent to saying that either $\mu^*_i = 0$ or $g_i(x^*) = 0$ for all $i$.

\begin{tcolorbox}[title=Slater's condition (Proposition~5.3.1)]
  Consider the convex program
  \[
  z = \inf\{f(x) : x \in X, g(x) \le 0\}
  \]
  where $X \subset \R^n$ is a convex set and $f, g_i : X \to \R$ are convex functions.

  The Lagrangian is
  \[
  L(x, \mu) &= f(x) + \mu^\tp g(x).
  \]

  Assume that $z$ is finite and that Slater's condition holds, meaning the set $\{x \in X : g(x) < 0\}$ is nonempty.
  Then strong duality holds, meaning
  \[z &= \sup_{\mu \ge 0} \inf_{x \in X} L(x, \mu).\]
  Furthermore the set of dual optimal solutions $\mu$ is nonempty and compact.
\end{tcolorbox}

\begin{tcolorbox}[title=Relaxed Slater's condition (Proposition~5.3.6)]
  Consider the convex program
  \[
  z = \inf\{f(x) : x \in P \cap C, g(x) \le 0, \tilde g(x) \le 0\}
  \]
  where:
  \begin{itemize}
  \item $P$ is polyhedral and $C$ is convex
  \item $f, g_i : C \to \R$ are convex
  \item Each $\tilde g_i : P \cap C \to \R$ is polyhedral (a maximum of a finite number of affine functions).
  \end{itemize}

  The Lagrangian is
  \[
  L(x, \mu, \tilde\mu) &= f(x) + \mu^\tp g(x) + \tilde\mu^\tp \tilde g(x).
  \]

  Assume that $z$ is finite and that the Relaxed Slater's condition holds, meaning the sets
  \[
  &P \cap \relint(C) \cap \{x : \tilde g(x) \le 0\} \\
  &P \cap C \cap \{x : g(x) < 0, \tilde g(x) \le 0\}
  \]
  are both nonempty.
  Then strong duality holds, meaning
  \[z &= \sup_{\mu, \tilde\mu \ge 0} \inf_{x \in P \cap C} L(x, \mu, \tilde\mu).\]
  Furthermore there exists a dual optimal solution $\mu, \tilde\mu$.
\end{tcolorbox}
\xxx{Can we say that the set of optimal $\mu$ is compact?}

\begin{itemize}
\item Proposition~5.3.7: An l.s.c. and compactness condition guarantees strong duality and that the set of primal optimal solutions is nonempty and compact.
\item Note that strong duality may not hold even if $f, g$ are convex proper continuous and $X$ is closed.
  For instance, consider $X = \{(x, y) \in \R^2 : x \ge 1\}, f(x, y) = 1 - y, g(x, y) = \frac{y^2}{x}$.
\end{itemize}

\section*{Minimax}

Let $\phi : X \times Z \to \R$ where $X \subset \R^n, Z \subset \R^n$.

\[
\phi_x(z) &\eqdef \phi(x, z) \\
p(u) &\eqdef \inf_x \sup_z \{\phi(x, z) - u^\tp z\} \\
&= \inf_x (-\phi_x)^*(-u) \\
q(\mu) &\eqdef -p^*(-\mu) \\
&= \inf_u \inf_x \sup_z \{\phi(x, z) - u^\tp z + u^\tp \mu\} \\
&= \inf_x \{-(-\phi_x)^{**}(\mu)\}
\]
\[
\marginnote{minimax inequality}
\sup_z \inf_x \phi(x, z)
\le \sup_z \inf_x \{-(-\phi_x)^{**}(z)\}
= p^{**}(0)
\le p(0)
&= \inf_x \sup_z \phi(x, z) \\
&= \inf_x \sup_z \{ (\ccvcl \phi_x)(z) \} \\
\]

\begin{itemize}
\item If $-\ccvcl \phi_x$ is proper for each $x$, then we have $-(-\phi_x)^{**} = \ccvcl \phi_x$.
\item If additionally $\phi_x = \ccvcl \phi_x$ for each $x$, then
  the minimax equality $\inf_x \sup_z \phi(x, z) = \sup_z \inf_x \phi(x, z)$ becomes equivalent to the strong duality inequality $p^{**}(0) \le p(0)$.
\end{itemize}

\printbibliography

\end{document}
