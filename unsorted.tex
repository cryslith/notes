\documentclass{article}
\usepackage[utf8]{inputenc}
\usepackage[margin=1in]{geometry}
\input{liatex/preamble}

\title{Unsorted notes}
\author{Lily Chung}
\date{}

\begin{document}
\maketitle

\begin{lemma}\label{thm:combi}
  \[\binom{n}{k} \le n^k\]
\end{lemma}
\begin{proof}
  Follows by combinatorial interpretation.
\end{proof}

The following is useful for combining multiplicative approximations.
\begin{lemma}
  If $x_i \in \left[1 - \frac{\epsilon}{n}, 1 + \frac{\epsilon \ln 2}{n}\right]$ where $\epsilon \in [0, 1]$ then
  \[\prod_{i \in [n]} x_i \in [1 - \epsilon, 1 + \epsilon]\]
\end{lemma}
\begin{proof}
  \[
  \prod_{i \in [n]} x_i &\le \left(1 + \frac{\epsilon \ln 2}{n}\right)^n \\
  &\le e^{\epsilon \ln 2} \\
  &= 2^\epsilon \\
  &\le 1 + \epsilon &\text{by Bernoulli's inequality} \\
  \prod_{i \in [n]} x_i &\ge \left(1 - \frac{\epsilon}{n}\right)^n \\
  &\ge 1 - \epsilon &\text{by Bernoulli's inequality}
  \]
\end{proof}

\end{document}
